\documentclass{article}
\usepackage[colorlinks,allcolors=blue,pagebackref=true]{hyperref}
\usepackage{graphicx} % Required for inserting images
\usepackage{natbib}
\usepackage{doi}
\usepackage{amssymb}
\usepackage{authblk}
\usepackage{enumitem}
\usepackage{cleveref}

\newcommand\WhileCC{\textit{\textbf{WhileCC}}}
\newcommand\Reals{\mathbb{R}}


\title{\WhileCC-Approximability and Acceptability of Elementary Functions}
\author[ ]{Fateme Ghasemi}
\author[ ]{Jeffery Zucker}
\author[ ]{W. F. Smyth}
\affil[ ]{Department of Computing and Software, McMaster University, Hamilton, ON, Canada}
\affil[ ]{ghases5@mcmaster.ca \qquad zucker@mcmaster.ca}

\date{January 2026}

\begin{document}

\maketitle

Computability theory on the reals is classically  developed for total functions \cite{ConcreteModelsTopologicalAlgebra_Tucker1999,ComputableTotalFunctionsOnMetricAlgebras_JohnTuckerAndJeffZucker}.
However, many standard functions in real analysis (such as the logarithmic, square root, and inverse trigonometric functions) are partial, and cannot be studied under such models.
Thus, in this work, we focus on the elementary functions \cite{OrdinaryDifferentialEquations_Morris_Tenenbaum_1985},
a subclass of single-valued partial functions on the reals
that is closed under
addition, multiplication, subtraction, division, n-th roots,
exponential, trigonometric functions, and their inverses.


Existing work \cite{ModelOfCompForPartFunc_MingQuanFuAndJeffZucker,ComputationByWhileProgramsOnTopologicalPartialAlgebras_TuckerAndZucker,AbstractVSConcreteComputationOnMetricPartialAlgebras_TuckerZucker_2004} studies four models of computation for partial functions on $\Reals$, namely
\begin{enumerate}
    \item GL-computability,
    \item tracking computability,
    \item multipolynomial
          approximability, and
    \item \WhileCC-approximability.
\end{enumerate}
The first two models are concrete models of computation, in which computability depends on the representation of data.
For example, in tracking computability, each real number is represented as a natural number.
In contrast, abstract models (such as \WhileCC-approximability) allow functions to be defined independently of such implementation details.
For a programmer, this would be akin to writing programs against an abstract interface instead of dealing with specific implementations.

Previous work by Fu and Zucker \cite{ModelOfCompForPartFunc_MingQuanFuAndJeffZucker} proves that these four models are equivalent when restricted to a class of functions we call ``acceptable''.
Intuitively, a function is acceptable if
\begin{enumerate}
    \item
          its domain is the union of a computable sequence of rational intervals called an \emph{effective open exhaustion}, and
    \item it statisfies a specific type of continuity called \emph{effective local uniform continuity}.
\end{enumerate}
Fu and Zucker's theorem means that, within the realm of acceptable functions, we can work with \WhileCC-approximability without giving up expressiveness and transfer results amongst the four models.

A natural question remained open: is the class of acceptable functions sufficiently large to include many common functions, such as the elementary functions?
In this work, we solve a conjecture posed by \cite{ModelOfCompForPartFunc_MingQuanFuAndJeffZucker} that all elementary functions are acceptable.
We also prove that the elementary functions are \WhileCC-approximable and therefore computable in all the aforementioned models of computation. Thus
% [Contributions]
% \paragraph{Contributions.}
the main contribution of this work is a proof of the acceptability of elementary functions (\Cref{sec::acceptability_of_elem_functions}, \Cref{thm::elem_funcs_are_acceptable}) derived from:
\begin{enumerate}
    \item a proof sketch for \WhileCC-approximability of all elementary functions;
    \item an explicit construction of effective open exhaustions for the domains of elementary functions;
    \item an alternative characterization of effective local uniform continuity (i.e. having a local continuity witness);
    \item a proof sketch that elementary functions have local continuity witnesses.
\end{enumerate}

\bibliographystyle{abbrvnat}
\bibliography{refs}
\end{document}
